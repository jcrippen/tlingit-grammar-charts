%!TEX root = ../grammar-charts.tex
%%
%% Aspects.
%%

\clearpage
\section{Aspects}\label{sec:asp}

\begin{multicols}{2}
\noindent
\end{multicols}

\clearpage
\begin{table}
\centerfloat
\begin{tabular}{llllcccccl}
\toprule
\textit{Aspect type}			& \textit{Asp.\ pfx.}	&\textit{Stv.\ pfx.}	& \textit{Rep.\ sfx.}	& \textit{Pfv.}
															& \textit{Stv.}
																& \textit{Dur.}
																	& \textit{Iter.}
																		& \textit{Lex.}
																			& \textit{Notes}\\
\midrule
lexical imperfective activity		&			&			&			& −	& −	& + 	& − 	& + 	&\\
lexical imperfective state		&			& i-			&			& −	& +	& ?	& −	& +	&\\
repetitive imperfective activity		&			&			& any			& −	& −	& +	& +	& ±	&\\
repetitive imperfective state		&			& i-			& -k			& −	& +	& ?	& +	& ±	&\\
\addlinespace[0.5em]
(unknown)				& \xx{cnj}-		&			&			& −	& −	& +	& −	& ?	&\\
conjugation-marked imperfective state	& \xx{cnj}-		& i-			&			& −	& +	& ?	& −	& +	& certain dimension-denoting roots\\
(unknown)				& \xx{cnj}-		&			& (unkn.)		& −	& −	& +	& +	& ?	&\\
multipositional imperfective state	& \xx{cnj}-		& i-			& -kw(-t)		& −	& +	& ?	& +	& +	&\\
\addlinespace[0.75em]
(unknown)				& wu-			&			&			& +	& −	& ?	& −	& −	&\\
perfective				& u- \~\ wu-		& i-			&			& +	& +	& ?	& −	& −	& \fm{u-} with some \fm{∅}-conjugation\\
habitual				& u-			&			& -ch			& +	& −	& ?	& +	& −	& \fm{∅}-conjugation only\\
repetitive perfective			& wu-			& i-			& any?			& +	& +	& ?	& +	& −	&\\
\addlinespace[0.5em]
(unknown)				& \xx{cnj}-		&			&			& +	& −	& ?	& −	& −	&\\
realizational				& \xx{cnj}-		& i-			&			& +	& +	& ?	& −	& −	&\\
habitual				& \xx{cnj}-		&			& -ch			& +	& −	& ?	& +	& −	& non-\fm{∅}-conjugation only\\
(unknown)				& \xx{cnj}-		& i-			& (unkn.)		& +	& +	& ?	& +	& −	&\\
\bottomrule
\end{tabular}
\caption{Imperfective and perfective aspect patterns}
\label{tab:aspect-patterns}
\end{table}

\clearpage
\subsection{Imperfective}\label{sec:asp-impfv}

\begin{multicols}{2}
\noindent
The imperfective aspect is mostly indicated by the lack of an overt aspectual prefix – no perfective \fm{(w)u-} and no conjugation \fm{n-}, \fm{g̱-}, or \fm{g-} – along with particular stem variation patterns.
Stem variation for most imperfective aspect forms is descriptively unpredictable.
This means that each root that supports an imperfective aspect form must specify the stem variation of that form.
There are some hints of correspondences between stem variation patterns and root meaning but their identification is still vague and unreliable.

All imperfective forms can be affirmative or negative.
When negative they contain an overt \fm{u-} irrealis prefix.
Some affirmative imperfective forms also contain a lexically or derivationally specified irrealis prefix.

Imperfective aspect forms are divided between activities and states by both morphological and semantic properties.
\textbf{Imperfective state} forms have an overt \fm{i-} stative prefix in affirmative main clauses and denote a state that holds at the reference time.
\textbf{Imperfective activity} forms do not have a stative prefix in affirmative main clauses and denote an activity that is ongoing at the reference time.
Imperfective state forms lose their \fm{i-} prefix in the same contexts as perfective aspect forms, i.e.\ with negation, with irrealis-marked dubitatives, with prohibitives and optatives, and in clauses with subordinate \fm{-í} \parencite[218–228]{crippen:2019}.

Among the imperfective states are some that have an overt conjugation prefix, i.e.\ one of \fm{n-}, \fm{g̱-}, or \fm{g-}.
This occurs only with roots that denote dimensions; the conjugation prefix reflects the usual spatial orientation of the conjugation classes: \fm{n-} lateral, \fm{g̱-} down, \fm{g-} up.
Thus \fm{g̱aadlaan} ‘it is deep’ is an affirmative imperfective state form of the root \fm{\rt[¹]{dlan}} ‘deep’ with the \fm{g̱-} conjugation prefix reflecting downward spatial orientation.
\end{multicols}

\clearpage
\begin{table}
\centerfloat
\begin{tabular}{l
		c@{\hspace{1ex}}c@{\hspace{1ex}}c@{\hspace{1ex}}c
		rrr
		*{5}{l}ll}
\toprule
\textit{Aspect}		& \textit{∅}
			    & \textit{n}
			        & \textit{g̱}
			            & \textit{g}
					& \textit{Preverbs}	& \textit{Asp.\ pfxs.}
										& \textit{Stv.}
											& \rt{CV}	& \rt{CVʰ}	& \rt{CVC}	& \rt{CVCʼ}	& \rt{CVʼC}	& \textit{Suffixes}	
																						& \textit{Notes}\\
\midrule

aff.\ impfv.\ act.	&   &   &   &   &			&		&	& -μμH		& -μμH		& -μμH		& -μμH		& -μμH		&	& Leer \fm{-·}\\
neg.\ impfv.\ act.	&   &   &   &   &			& u-		&	& -μμL		& -μμL		& -μμL		& -μμH		& -μμH		&	& Leer \fm{-ʻ}\\
\addlinespace[0.5em]
aff.\ impfv.\ act.	&   &   &   &   &			&		&	& -μμL		& -μμL		& -μμL		& -μμH		& -μμH		&	& Leer \fm{-ʻ}\\
neg.\ impfv.\ act.	&   &   &   &   &			& u-		&	& -μμL		& -μμL		& -μμL		& -μμH		& -μμH		&	& Leer \fm{-ʻ}\\
\addlinespace[0.5em]
aff.\ impfv.\ act.	&   &   &   &   &			&		&	& -μH		& -μH		&		&		&		&	& Leer \fm{-ʼ}\\
neg.\ impfv.\ act.	&   &   &   &   &			& u-		&	& -μH		& -μH		&		&		&		&	& Leer \fm{-ʼ}\\
\addlinespace[0.5em]
aff.\ impfv.\ act.	&   &   &   &   &			&		&	&		&		& -μH		& -μH		& -μH		&	& Leer \fm{-n}\\
neg.\ impfv.\ act.	&   &   &   &   &			& u-		&	&		&		& -μH		& -μH		& -μH		&	& Leer \fm{-n}\\
\bottomrule
\end{tabular}
\caption{Aspect morphology: imperfective aspect – lexical activity}
\label{tab:aspect-morphology-impfv-act}
\end{table}

\begin{table}
\centerfloat
\begin{tabular}{l
		c@{\hspace{1ex}}c@{\hspace{1ex}}c@{\hspace{1ex}}c
		rrr
		*{5}{l}ll}
\toprule
\textit{Aspect}		& \textit{∅}
			    & \textit{n}
			        & \textit{g̱}
			            & \textit{g}
					& \textit{Preverbs}	& \textit{Asp.\ pfxs.}
										& \textit{Stv.}
											& \rt{CV}	& \rt{CVʰ}	& \rt{CVC}	& \rt{CVCʼ}	& \rt{CVʼC}	& \textit{Suffixes}	
																						& \textit{Notes}\\
\midrule

aff.\ impfv.\ state	&   &   &   &   &			&		& i-	& -μμH		& -μμH		& -μH		& -μH		& -μH		&	& Leer \fm{-ÿ}\\
neg.\ impfv.\ state	&   &   &   &   &			& u-		&	& -μH		& -μH		& -μH		& -μH		& -μH		&	& Leer \fm{-ʼ/ÿ}\\
\addlinespace[0.5em]
aff.\ impfv.\ state	&   &   &   &   &			&		& i-	& -μμL		& -μμL		& -μμL		& -μμH		& -μμH		&	& Leer \fm{-ʻ}\\
neg.\ impfv.\ state	&   &   &   &   &			& u-		&	& -μH		& -μH		& -μμL		& -μμH		& -μμH		&	& Leer \fm{-ʼ/ʻ}\\
\addlinespace[0.5em]
aff.\ impfv.\ state	&   &   &   &   &			&		& i-	& -μμH		& -μμH		& -μμH		& -μμH		& -μμH		&	& Leer \fm{-·}\\
neg.\ impfv.\ state	&   &   &   &   &			& u-		&	& -μμL		& -μμL		& -μμL		& -μμH		& -μμH		&	& Leer \fm{-ʻ}\\
\addlinespace[0.5em]
aff.\ impfv.\ state	&   &   &   &   &			&		& i-	&		& -eμH		&		&		&		& -n	& \fm{\rt[¹]{haʰ}} ‘many’\\
neg.\ impfv.\ state	&   &   &   &   &			& u-		&	&		& -eμH		&		&		&		& -n	&\\
\bottomrule
\end{tabular}
\caption{Aspect morphology: imperfective aspect – lexical state}
\label{tab:aspect-morphology-impfv-stv}
\end{table}

\begin{table}
\centerfloat
\begin{tabular}{l
		c@{\hspace{1ex}}c@{\hspace{1ex}}c@{\hspace{1ex}}c
		rrr
		*{5}{l}ll}
\toprule
\textit{Aspect}		& \textit{∅}
			    & \textit{n}
			        & \textit{g̱}
			            & \textit{g}
					& \textit{Preverbs}	& \textit{Asp.\ pfxs.}
										& \textit{Stv.}
											& \rt{CV}	& \rt{CVʰ}	& \rt{CVC}	& \rt{CVCʼ}	& \rt{CVʼC}	& \textit{Suffixes}	
																						& \textit{Notes}\\
\midrule

aff.\ impfv.\ state	&   & ✓ &   &   &			& n-		& i-	& -μμL		&		&		&		&		&	& \fm{\rt[¹]{da}} ‘flow’\\
neg.\ impfv.\ state	&   & ✓ &   &   &			& u-n-		&	& -μμL		&		&		&		&		&	&\\
\addlinespace[0.5em]
aff.\ impfv.\ state	&   &   & ✓ &   &			& g̱-		& i-	& -μμL		&		& -μμL		&		&&& \fm{\rt[¹]{dlan}} ‘deep’, \fm{\rt[¹]{da}} ‘flow’\\
neg.\ impfv.\ state	&   &   & ✓ &   &			& u-g̱-		&	& -μμL		&		& -μμL		&		&		&	&\\
\addlinespace[0.5em]
aff.\ impfv.\ state	&   &   & ✓ &   &			& g̱-		& i-	&		&		& -μμH		&		&		&	& \fm{\rt[¹]{tsʼan}} ‘shallow’\\
neg.\ impfv.\ state	&   &   & ✓ &   &			& u-g̱-		&	&		&		& -μμL		&		&		&	& \\
\addlinespace[0.5em]
aff.\ impfv.\ state	&   & ✓ &   &   &			& n-		& i-	&		& -μμH		&		&		&&& \fm{\rt[¹]{leʰ}} ‘far’, \fm{\rt[¹]{seʰ}} ‘near’,\\
neg.\ impfv.\ state	&   & ✓ &   &   &			& u-n-		&	&		& -μH		&		&		&		&& \&\ \fm{\rt[¹]{shuʰ}} ‘extend’\\
\addlinespace[0.5em]
aff.\ impfv.\ state	&   &   & ✓ &   &			& g̱-		& i-	&		& -μμH		&		&		&		&	&\fm{\rt[¹]{shuʰ}} ‘extend’\\
neg.\ impfv.\ state	&   &   & ✓ &   &			& u-g̱-		&	&		& -μH		&		&		&		&	&\\
\addlinespace[0.5em]
aff.\ impfv.\ state	&   &   &   & ✓ &			& g-		& i-	&		& -μμH		&		&		&		&	&\fm{\rt[¹]{shuʰ}} ‘extend’\\
neg.\ impfv.\ state	&   &   &   & ✓ &			& g-u-		&	&		& -μH		&		&		&		&	&\\
\bottomrule
\end{tabular}
\caption{Aspect morphology: imperfective aspect – conjugation-marked extended lexical state}
\label{tab:aspect-morphology-impfv-xtn}
\end{table}


\begin{table}
\centerfloat
\begin{tabular}{l
		c@{\hspace{1ex}}c@{\hspace{1ex}}c@{\hspace{1ex}}c
		rrr
		*{5}{l}ll}
\toprule
\textit{Aspect}		& \textit{∅}
			    & \textit{n}
			        & \textit{g̱}
			            & \textit{g}
					& \textit{Preverbs}	& \textit{Asp.\ pfxs.}
										& \textit{Stv.}
											& \rt{CV}	& \rt{CVʰ}	& \rt{CVC}	& \rt{CVCʼ}	& \rt{CVʼC}	& \textit{Suffixes}	
																						& \textit{Notes}\\
\midrule

aff.\ impfv.\ state	&   & ✓ &   &   &			& n-		& i-	& -μH		&		&		&		&		& -kw-t	& \fm{\rt[¹]{da}} ‘flow’\\
neg.\ impfv.\ state	&   & ✓ &   &   &			& u-n-		&	& -μH		&		&		&		&		& -kw-t	&\\
\addlinespace[0.5em]
aff.\ impfv.\ state	&   & ✓ &   &   &			& n-		& i-	&		&		& -μH		&		&		& -k	& \fm{\rt[¹]{.at}} ‘pl.\ go’\\
neg.\ impfv.\ state	&   & ✓ &   &   &			& u-n-		&	&		&		& -μH		&		&		& -k	&\\
\bottomrule
\end{tabular}
\caption{Aspect morphology: imperfective aspect – conjugation-marked multipositional repetitive state}
\label{tab:aspect-morphology-impfv-mult-rep}
\end{table}

\begin{table}
\centerfloat
\begin{tabular}{l
		c@{\hspace{1ex}}c@{\hspace{1ex}}c@{\hspace{1ex}}c
		rrr
		*{5}{l}ll}
\toprule
\textit{Aspect}		& \textit{∅}
			    & \textit{n}
			        & \textit{g̱}
			            & \textit{g}
					& \textit{Preverbs}	& \textit{Asp.\ pfxs.}
										& \textit{Stv.}
											& \rt{CV}	& \rt{CVʰ}	& \rt{CVC}	& \rt{CVCʼ}	& \rt{CVʼC}	& \textit{Suffixes}	
																						& \textit{Notes}\\
\midrule

aff.\ rep.\ impfv.	& ✓ &   &   &   &			& 		&	& -μͤμH		& -μͤμL		& -μH		& -μH		& -μH		& -x̱	&\\
neg.\ rep.\ impfv.	& ✓ &   &   &   &			& u-		&	& -μͤμH		& -μͤμL		& -μH		& -μH		& -μH		& -x̱	&\\
\addlinespace[0.5em]
aff.\ rep.\ impfv.	&   & ✓ &   &   & yoo=			& 		& i-	& -μͤμH		& -μͤμL		& -μH		& -μH		& -μH		& -k	&\\
neg.\ rep.\ impfv.	&   & ✓ &   &   & yoo=			& u-		&	& -μͤμH		& -μͤμL		& -μH		& -μH		& -μH		& -k	&\\
\addlinespace[0.5em]
aff.\ rep.\ impfv.	&   &   & ✓ &   & yei=			& 		&	& -μͤμH		& -μͤμL		& -μH		& -μH		& -μH		& -ch	&\\
neg.\ rep.\ impfv.	&   &   & ✓ &   & yei=			& u-		&	& -μͤμH		& -μͤμL		& -μH		& -μH		& -μH		& -ch	&\\
\addlinespace[0.5em]
aff.\ rep.\ impfv.	&   &   &   & ✓ & kei=			& 		&	& -μͤμH		& -μͤμL		& -μH		& -μH		& -μH		& -ch	&\\
neg.\ rep.\ impfv.	&   &   &   & ✓ & kei=			& u-		&	& -μͤμH		& -μͤμL		& -μH		& -μH		& -μH		& -ch	&\\
\bottomrule
\end{tabular}
\caption{Aspect morphology: imperfective aspect – conjugation class–determined repetitive}
\label{tab:aspect-morphology-impfv-cnj-rep}
\end{table}

\clearpage
\subsection{Perfective}\label{sec:asp-pfv}

\begin{multicols}{2}
\noindent
\end{multicols}

\clearpage
\begin{table}
\centerfloat
\begin{tabular}{l
		c@{\hspace{1ex}}c@{\hspace{1ex}}c@{\hspace{1ex}}c
		rrr
		*{5}{l}ll}
\toprule
\textit{Aspect}		& \textit{∅}
			    & \textit{n}
			        & \textit{g̱}
			            & \textit{g}
					& \textit{Preverbs}	& \textit{Asp.\ pfxs.}
										& \textit{Stv.}
											& \rt{CV}	& \rt{CVʰ}	& \rt{CVC}	& \rt{CVCʼ}	& \rt{CVʼC}	& \textit{Suffixes}	
																						& \textit{Notes}\\
\midrule
aff.\ perfective	& ✓ &   &   &   &			& (w)u-		& i-	& -μμH		& -μμH		& -μH		& -μH		& -μH		&	&\\
			&   & ✓ & ✓ & ✓	&			& wu-		& i-	& -μμL		& -μμL		& -μμL		& -μμH		& -μμH		&	&\\
\addlinespace[0.5em]
neg.\ perfective	& ✓ &   &   &   &			& wu-		&	& -μH		& -μH		& -μμL		& -μμH		& -μμH		&	&\\
			&   & ✓ & ✓ & ✓ &			& wu-		&	& -μμL		& -μμL		& -μμL		& -μμH		& -μμH		&	&\\
\bottomrule
\end{tabular}
\caption{Aspect morphology: perfective aspect \textit{u-} \~\ \textit{wu-}}
\label{tab:aspect-morphology-pfv}
\end{table}

\clearpage
\subsection{Habitual}\label{sec:asp-hab}

\begin{multicols}{2}
\noindent
\end{multicols}

\clearpage
\begin{table}
\centerfloat
\begin{tabular}{l
		c@{\hspace{1ex}}c@{\hspace{1ex}}c@{\hspace{1ex}}c
		rrr
		*{5}{l}ll}
\toprule
\textit{Aspect}		& \textit{∅}
			    & \textit{n}
			        & \textit{g̱}
			            & \textit{g}
					& \textit{Preverbs}	& \textit{Asp.\ pfxs.}
										& \textit{Stv.}
											& \rt{CV}	& \rt{CVʰ}	& \rt{CVC}	& \rt{CVCʼ}	& \rt{CVʼC}	& \textit{Suffixes}	
																						& \textit{Notes}\\
\midrule
aff.\ habitual		& ✓ &   &   &   &			& (w)u-		&	& -μμH		& -μμH		& 		& 		& 	& \llap{-ÿ}-ch	& no stem ablaut for \fm{a/u}\\
			& ✓ &   &   &   &			& (w)u-		&	&		&		& -μH		& -μH		& -μH		& -ch	& \\
			& ✓ &   &   &   & \xx{dir}=\pr{D}	& (w)u-		&	&		&		& -μμL		& -μμH		& -μμH		& -ch	& motion with grp.\ D preverbs\\
\addlinespace[0.25em]
			&   & ✓ &   &   &			& n-		&	& -μͤμH		& -μͤμL		& -μH		& -μH		& -μH		& -ch	&\\
			&   &   & ✓ &   &			& g̱-		&	& -μͤμH		& -μͤμL		& -μH		& -μH		& -μH		& -ch	&\\
			&   &   &   & ✓ &			& g-		&	& -μͤμH		& -μͤμL		& -μH		& -μH		& -μH		& -ch	&\\
\addlinespace[0.75em]
neg.\ habitual		& ✓ &   &   &   &			& (w)u-		&	& -μμH		& -μμH		& 		& 		& 	& \llap{-ÿ}-ch	& no stem ablaut for \fm{a/u}\\
			& ✓ &   &   &   &			& (w)u-		&	&		&		& -μH		& -μH		& -μH		& -ch	& \\
			& ✓ &   &   &   & \xx{dir}=\pr{D}	& (w)u-		&	&		&		& -μμL		& -μμH		& -μμH		& -ch	& motion with grp.\ D preverbs\\
\addlinespace[0.25em]
			&   & ✓ &   &   &			& u-n-		&	& -μͤμH		& -μͤμL		& -μH		& -μH		& -μH		& -ch	&\\
			&   &   & ✓ &   &			& u-g̱-		&	& -μͤμH		& -μͤμL		& -μH		& -μH		& -μH		& -ch	&\\
			&   &   &   & ✓ &			& g-u-		&	& -μͤμH		& -μͤμL		& -μH		& -μH		& -μH		& -ch	&\\
\bottomrule
\end{tabular}
\caption{Aspect morphology: habitual aspect \textit{u-}/\textit{\xx{cnj-}} … \fm{-ch}}
\label{tab:aspect-morphology-hab}
\end{table}

\clearpage
\subsection{Progressive}\label{sec:asp-prog}

\begin{multicols}{2}
\noindent
\end{multicols}

\clearpage
\begin{table}
\centerfloat
\begin{tabular}{l
		c@{\hspace{1ex}}c@{\hspace{1ex}}c@{\hspace{1ex}}c
		rrr
		*{5}{l}ll}
\toprule
\textit{Aspect}		& \textit{∅}
			    & \textit{n}
			        & \textit{g̱}
			            & \textit{g}
					& \textit{Preverbs}	& \textit{Asp.\ pfxs.}
										& \textit{Stv.}
											& \rt{CV}	& \rt{CVʰ}	& \rt{CVC}	& \rt{CVCʼ}	& \rt{CVʼC}	& \textit{Suffixes}	
																						& \textit{Notes}\\
\midrule
aff.\ progressive	& ✓ & ✓ &   &   & ÿaa=			& n-		&	& -μͤμH		& -μͤμH		&		&		&		& -n	&\\
			& ✓ & ✓ &   &   & ÿaa=			& n-		&	&		&		& -μH		& -μH		& -μH	& \llap{(}-n)	& usually no \fm{-n}\\
\addlinespace[0.25em]
			&   &   & ✓ &   & yei=			& n-		&	& -μͤμH		& -μͤμH		&		&		&		& -n	&\\
			&   &   & ✓ &   & yei=			& n-		&	&		&		& -μH		& -μH		& -μH	& \llap{(}-n)	& usually no \fm{-n}\\
\addlinespace[0.25em]
			&   &   &   & ✓ & kei=			& n-		&	& -μͤμH		& -μͤμH		&		&		&		& -n	&\\
			&   &   &   & ✓ & kei=			& n-		&	&		&		& -μH		& -μH		& -μH	& \llap{(}-n)	& usually no \fm{-n}\\
\addlinespace[0.75em]
neg.\ progressive	& ✓ & ✓ &   &   & ÿaa=			& u-n-		&	& -μͤμH		& -μͤμH		&		&		&		& -n	&\\
			& ✓ & ✓ &   &   & ÿaa=			& u-n-		&	&		&		& -μH		& -μH		& -μH	& \llap{(}-n)	& usually no \fm{-n}\\
\addlinespace[0.25em]
			&   &   & ✓ &   & yei=			& u-n-		&	& -μͤμH		& -μͤμH		&		&		&		& -n	&\\
			&   &   & ✓ &   & yei=			& u-n-		&	&		&		& -μH		& -μH		& -μH	& \llap{(}-n)	& usually no \fm{-n}\\
\addlinespace[0.25em]
			&   &   &   & ✓ & kei=			& u-n-		&	& -μͤμH		& -μͤμH		&		&		&		& -n	&\\
			&   &   &   & ✓ & kei=			& u-n-		&	&		&		& -μH		& -μH		& -μH	& \llap{(}-n)	& usually no \fm{-n}\\
\bottomrule
\end{tabular}
\caption{Aspect morphology: progressive aspect \textit{n-} … \textit{-n}}
\label{tab:aspect-morphology-prog}
\end{table}

\clearpage
\subsection{Prospective}\label{sec:asp-prosp}

\begin{multicols}{2}
\noindent
\end{multicols}

\clearpage
\begin{table}
\centerfloat
\begin{tabular}{l
		c@{\hspace{1ex}}c@{\hspace{1ex}}c@{\hspace{1ex}}c
		rrr
		*{5}{l}ll}
\toprule
\textit{Aspect}		& \textit{∅}
			    & \textit{n}
			        & \textit{g̱}
			            & \textit{g}
					& \textit{Preverbs}	& \textit{Asp.\ pfxs.}
										& \textit{Stv.}
											& \rt{CV}	& \rt{CVʰ}	& \rt{CVC}	& \rt{CVCʼ}	& \rt{CVʼC}	& \textit{Suffixes}	
																						& \textit{Notes}\\
\midrule
aff.\ prospective	& ✓ & ✓ &   &   &			& g-w-g̱-	&	& -μμH		& -μμH		& -μμH		& -μμH		& -μμH		&	&\\
			&   &   & ✓ &   & yei=			& g-w-g̱-	&	& -μμH		& -μμH		& -μμH		& -μμH		& -μμH		&	&\\
			&   &   &   & ✓ & kei=			& g-w-g̱-	&	& -μμH		& -μμH		& -μμH		& -μμH		& -μμH		&	&\\
\addlinespace[0.5em]
neg.\ prospective	& ✓ & ✓ &   &   &			& g-w-g̱-	&	& -μμL		& -μμL		& -μμL		& -μμL		& -μμL		&	&\\
			&   &   & ✓ &   & yei=			& g-w-g̱-	&	& -μμL		& -μμL		& -μμL		& -μμL		& -μμL		&	&\\
			&   &   &   & ✓ & kei=			& g-w-g̱-	&	& -μμL		& -μμL		& -μμL		& -μμL		& -μμL		&	&\\
\bottomrule
\end{tabular}
\caption{Aspect morphology: prospective aspect \textit{g-} + \textit{w-} + \textit{g̱-}}
\label{tab:aspect-morphology-prosp}
\end{table}

\clearpage
\subsection{Imperative}\label{sec:asp-imp}

\begin{multicols}{2}
\noindent
Imperatives are indicated by the presence of the lexically or derivationally specified conjugation prefix, no other aspectual prefixes, no stative prefix, and specific stem variation.
The stem variation of a particular form is dependent on both root shape and conjugation class.
For the \fm{n-}, \fm{g̱-}, and \fm{g}-conjugation classes the imperative stem is \fm{-μμL} except for \fm{\rt{CVCʼ}} and \fm{\rt{CVʼC}} roots where it is predictably \fm{-μμH} instead.
For the \fm{∅}-conjugation class the imperative stem is \fm{-μμL} for open syllable roots and \fm{-μμH} for closed syllable roots.

The description above is the usual case, but there are some unusual stem variation patterns.
One such pattern involves the \textbf{\fm{∅⁺}-conjugation verbs}, a group of open syllable (\fm{\rt{CV}}, \fm{\rt{CVʰ}}) roots forming verbs in the \fm{∅}-conjugation class that have unexpected stem variation in the imperative, hortative, and potential \parencites[204, 269–270]{leer:1991}[84–86]{eggleston:2013a}[155–156, 160, 163–164]{crippen:2019}.
The imperative forms of these particular roots have \fm{-μμH} stems rather than the usual \fm{-μμL} of other open syllable roots in the \fm{∅}-conjugation class.

Another unusual pattern is that closed syllable roots in the \fm{∅}-conjugation class have unusual stem variation when they occur as a \textbf{motion verb with certain preverbs}.
The specific preverbs are \fm{kei=} ‘up’, \fm{yei=} ‘down’, \fm{ÿeeḵ=} \~\ \fm{ÿeiḵ=} ‘beach’, \fm{daaḵ=} ‘inland’, \fm{daak=} ‘out to sea’, and \fm{g̱unaÿéi=} \~\ \fm{g̱unéi=} ‘beginning, starting’.
The imperative form of a closed syllable root with one of these preverbs will have \fm{-μμL} (with \fm{\rt{CVC}}) or \fm{-μμH} (with \fm{\rt{CVCʼ}} or \fm{\rt{CVʼC}}) stem variation instead of the usual \fm{-μH} \parencites[176]{story:1966}[209]{leer:1991}[86–87]{eggleston:2013a}[154–155]{crippen:2019}.
As can be seen in tables \ref{tab:preverbs-group-H} and \ref{tab:preverbs-group-D}, these are almost all group D preverbs except for \fm{g̱unaÿéi=} \~\ \fm{g̱unéi=} ‘beginning, starting’ which belongs to group H instead.
The imperative patterns in table \ref{tab:aspect-morphology-imp} represent this subset of motion verbs with the symbol \fm{\xx{pvb}=}.
\citeauthor{story:1966} says that the \fm{neil} ‘home, inside’ of the group E preverbs (table \ref{tab:preverbs-group-E}) also has the same unusual stem variation pattern \parencite[176]{story:1966}, but examples in \cite{eggleston:2017} have the usual \fm{-μH} that is expected for closed syllable roots in the \fm{∅}-conjugation class.
This may be a point of grammatical variation and it needs to be explored further.

The last unusual pattern is the unusual stems of any imperative verbs with one of the three roots \fm{\rt[¹]{gut}} ‘sg.\ go’, \fm{\rt[¹]{.at}} ‘pl.\ go; handle pl.’, or \fm{\rt[¹]{nuk}} ‘sg.\ sit down’ \parencites[49]{story:1966}[154–155]{crippen:2019}.
The imperative verb stems lack the coda consonant that appears in all other verbs based on these roots.
This is symbolized by \fm{-⊗} which iconically reflects the deletion of the final consonant.
The resulting stems have the shape \fm{–CV́}, i.e.\ an open syllable with a short vowel and high tone.
Because these are motion verb roots they may occur in any of the four conjugation classes, but their imperative stems in all four classes are identical.

Imperative forms are unique in that they \textbf{only occur with second person arguments}.
For transitives the subject must be second person but there is no constraint on the object.
For subject intransitives (unergatives) the subject must be second person.
For object intransitives (unaccusatives) the object must be second person.
The second person argument can be either singular or plural except where one of these is prohibited by the root (e.g.\ \fm{\rt[¹]{gut}} ‘sg.\ go’).
Although the second person singular subject is normally \fm{i-} ‘you sg.’, in imperatives it is absent (covert) when there is no \fm{d-} prefix; this is \textbf{imperative subject dropping}.
If there is a \fm{d-} prefix then the second person singular subject \fm{i-} is overt.
The second person plural subject \fm{ÿi-} ‘you pl.’\ is always overt.
These unique argument patterns of the imperative are shown in table \ref{tab:aspect-morphology-imp-args}.
\end{multicols}

\clearpage
\begin{table}
\centerfloat
\begin{tabular}{l
		c@{\hspace{1ex}}c@{\hspace{1ex}}c@{\hspace{1ex}}c
		rrr
		*{5}{l}ll}
\toprule
\textit{Aspect}		& \textit{∅}
			    & \textit{n}
			        & \textit{g̱}
			            & \textit{g}
					& \textit{Preverbs}	& \textit{Asp.\ pfxs.}
										& \textit{Stv.}
											& \rt{CV}	& \rt{CVʰ}	& \rt{CVC}	& \rt{CVCʼ}	& \rt{CVʼC}	& \textit{Suffixes}	
																						& \textit{Notes}\\
\midrule
imperative		& ✓ &   &   &	&	 		&		&	& -μμL		& -μμL		& -μH		& -μH		& -μH		&	&\\
			& ✓ &   &   &   &			& 		&	& -μμH		& -μμH		&		&		&		&	& \fm{∅⁺}-conj., \rt{CV⁽ʰ⁾} only\\
			& ✓ &   &   &   & \xx{pvb}=		& 		&	&		&		& -μμL		& -μμH		& -μμH		&	& motion with some preverbs\\
\addlinespace[0.5em]
			&   & ✓ &   &   &			& n-		&	& -μμL		& -μμL		& -μμL 		& -μμH		& -μμH		&	&\\
			&   &   & ✓ &   &			& g̱-		&	& -μμL		& -μμL		& -μμL		& -μμH		& -μμH		&	&\\
			&   &   &   & ✓ &			& g-		&	& -μμL		& -μμL		& -μμL		& -μμH		& -μμH		&	&\\
\addlinespace[0.5em]
			& ✓ & ✓ & ✓ & ✓ &			& \xx{cnj}-	&	&		&		& -⊗ \rlap{(–CV́)}&		&		&	& \fm{\rt[¹]{gut}} ‘go’, \fm{\rt[¹]{.at}} ‘go’, \fm{\rt[¹]{nuk}} ‘sit’\\
\bottomrule
\end{tabular}
\caption{Aspect morphology: imperative mood \textit{\xx{cnj}-}}
\label{tab:aspect-morphology-imp}
\end{table}

\begin{table}
\centerfloat
\begin{tabular}{lrrrrl}
\toprule
				& \textit{Asp.\ pfx.}
						& \textit{Subj.}
							& \textit{d-}
								& \textit{s-/l-/sh-}
										& \textit{Notes}\\
\midrule
\xx{2sg} subject		& \xx{cnj}-	&	&	&		&\\
				& \xx{cnj}-	&	& 	& s-/l-/sh-	&\\
				& \xx{cnj}-	& i-	& d-	&		&\\
				& \xx{cnj}-	& i-	& d-	& s-/l-/sh-	&\\
\addlinespace[0.5em]
\xx{2pl} subject		& \xx{cnj}-	& ÿi-	&	&		&\\
				& \xx{cnj}-	& ÿi-	&	& s-/l-/sh-	&\\
				& \xx{cnj}-	& ÿi-	& d-	&		&\\
				& \xx{cnj}-	& ÿi-	& d-	& s-/l-/sh-	&\\
\bottomrule
\end{tabular}
\caption{Subject patterns for imperative mood}
\label{tab:aspect-morphology-imp-args}
\end{table}

\clearpage
\subsection{Hortative and potential}\label{sec:asp-hortpot}

\begin{multicols}{2}
\noindent
\end{multicols}

\clearpage
\begin{table}
\centerfloat
\begin{tabular}{l
		c@{\hspace{1ex}}c@{\hspace{1ex}}c@{\hspace{1ex}}c
		rrr
		*{5}{l}ll}
\toprule
\textit{Aspect}		& \textit{∅}
			    & \textit{n}
			        & \textit{g̱}
			            & \textit{g}
					& \textit{Preverbs}	& \textit{Asp.\ pfxs.}
										& \textit{Stv.}
											& \rt{CV}	& \rt{CVʰ}	& \rt{CVC}	& \rt{CVCʼ}	& \rt{CVʼC}	& \textit{Suffixes}	
																						& \textit{Notes}\\
\midrule
hortative		& ✓ &   &   &   &			& g̱-		&	& -μμL		& -μμL		& -μμL		& -μμH		& -μμH		&	&\\
			& ✓ &   &   &   &			& g̱-		&	& -μμH		& -μμH		&		&		&		&	& \fm{∅⁺}-conj., \rt{CV⁽ʰ⁾} only\\
\addlinespace[0.25em]
			&   & ✓ &   &   &			& n-g̱-		&	& -μμL		& -μμL		& -μμL 		& -μμH		& -μμH		&	&\\
			&   &   & ✓ &   &			& g̱-g̱-		&	& -μμL		& -μμL		& -μμL		& -μμH		& -μμH		&	&\\
			&   &   &   & ✓ &			& g-g̱-		&	& -μμL		& -μμL		& -μμL		& -μμH		& -μμH		&	&\\
\bottomrule
\end{tabular}
\caption{Aspect morphology: hortative modality \textit{\xx{cnj}-} + \textit{g̱-}}
\label{tab:aspect-morphology-hort}
\end{table}

\begin{table}
\centerfloat
\begin{tabular}{l
		c@{\hspace{1ex}}c@{\hspace{1ex}}c@{\hspace{1ex}}c
		rrr
		*{5}{l}ll}
\toprule
\textit{Aspect}		& \textit{∅}
			    & \textit{n}
			        & \textit{g̱}
			            & \textit{g}
					& \textit{Preverbs}	& \textit{Asp.\ pfxs.}
										& \textit{Stv.}
											& \rt{CV}	& \rt{CVʰ}	& \rt{CVC}	& \rt{CVCʼ}	& \rt{CVʼC}	& \textit{Suffixes}	
																						& \textit{Notes}\\
\midrule
potential		& ✓ &   &   &   &			& u-g̱-		&	& -μμL		& -μμL		& -μμL		& -μμH		& -μμH		&	&\\
			& ✓ &   &   &   &			& u-g̱-		&	& -μμH		& -μμH		&		&		&		&	& \fm{∅⁺}-conj., \rt{CV⁽ʰ⁾} only\\
\addlinespace[0.25em]
			&   & ✓ &   &   &			& u-n-g̱-	&	& -μμL		& -μμL		& -μμL 		& -μμH		& -μμH		&	&\\
			&   &   & ✓ &   &			& u-g̱-g̱-	&	& -μμL		& -μμL		& -μμL		& -μμH		& -μμH		&	&\\
			&   &   &   & ✓ &			& g-u-g̱-	&	& -μμL		& -μμL		& -μμL		& -μμH		& -μμH		&	&\\
\bottomrule
\end{tabular}
\caption{Aspect morphology: potential modality \textit{u-} + \textit{\xx{cnj}-} + \textit{g̱-}}
\label{tab:aspect-morphology-pot}
\end{table}

\clearpage
\subsection{Realizational, admonitive, and consecutive}\label{sec:asp-rlznadmoncsec}

\begin{multicols}{2}
\noindent
\end{multicols}

\clearpage
\begin{table}
\centerfloat
\begin{tabular}{l
		c@{\hspace{1ex}}c@{\hspace{1ex}}c@{\hspace{1ex}}c
		rrr
		*{5}{l}ll}
\toprule
\textit{Aspect}		& \textit{∅}
			    & \textit{n}
			        & \textit{g̱}
			            & \textit{g}
					& \textit{Preverbs}	& \textit{Asp.\ pfxs.}
										& \textit{Stv.}
											& \rt{CV}	& \rt{CVʰ}	& \rt{CVC}	& \rt{CVCʼ}	& \rt{CVʼC}	& \textit{Suffixes}	
																						& \textit{Notes}\\
\midrule
realizational		& ✓ &   &   &   &			&		& i-	& -μμH		& -μμH		& -μμH		& -μμH		& -μμH		&	&\\
			&   & ✓ &   &   &			& n-		& i-	& -μμH		& -μμH		& -μμH 		& -μμH		& -μμH		&	&\\
			&   &   & ✓ &   &			& g̱-		& i-	& -μμH		& -μμH		& -μμH		& -μμH		& -μμH		&	&\\
			&   &   &   & ✓ &			& g-		& i-	& -μμH		& -μμH		& -μμH		& -μμH		& -μμH		&	&\\
\bottomrule
\end{tabular}
\caption{Aspect morphology: realizational aspect \textit{\xx{cnj}-} + \textit{i-}}
\label{tab:aspect-morphology-rlzn}
\end{table}

\begin{table}
\centerfloat
\begin{tabular}{l
		c@{\hspace{1ex}}c@{\hspace{1ex}}c@{\hspace{1ex}}c
		rrr
		*{5}{l}ll}
\toprule
\textit{Aspect}		& \textit{∅}
			    & \textit{n}
			        & \textit{g̱}
			            & \textit{g}
					& \textit{Preverbs}	& \textit{Asp.\ pfxs.}
										& \textit{Stv.}
											& \rt{CV}	& \rt{CVʰ}	& \rt{CVC}	& \rt{CVCʼ}	& \rt{CVʼC}	& \textit{Suffixes}	
																						& \textit{Notes}\\
\midrule
admonitive		& ✓ &   &   &   &			& u-		&	& -μμH		& -μμH		& -μμH		& -μμH		& -μμH		&	&\\
			&   & ✓ &   &   &			& u-n-		&	& -μμH		& -μμH		& -μμH 		& -μμH		& -μμH		&	&\\
			&   &   & ✓ &   &			& u-g̱-		&	& -μμH		& -μμH		& -μμH		& -μμH		& -μμH		&	&\\
			&   &   &   & ✓ &			& g-u-		&	& -μμH		& -μμH		& -μμH		& -μμH		& -μμH		&	&\\
\bottomrule
\end{tabular}
\caption{Aspect morphology: admonitive mood \textit{u-} + \textit{\xx{cnj}-}}
\label{tab:aspect-morphology-admon}
\end{table}

\begin{table}
\centerfloat
\begin{tabular}{l
		c@{\hspace{1ex}}c@{\hspace{1ex}}c@{\hspace{1ex}}c
		rrr
		*{5}{l}ll}
\toprule
\textit{Aspect}		& \textit{∅}
			    & \textit{n}
			        & \textit{g̱}
			            & \textit{g}
					& \textit{Preverbs}	& \textit{Asp.\ pfxs.}
										& \textit{Stv.}
											& \rt{CV}	& \rt{CVʰ}	& \rt{CVC}	& \rt{CVCʼ}	& \rt{CVʼC}	& \textit{Suffixes}	
																						& \textit{Notes}\\
\midrule
consecutive		& ✓ &   &   &   &			&		&	& -μμH		& -μμH		& -μμH		& -μμH		& -μμH		&	&\\
			&   & ✓ &   &   &			& n-		&	& -μμH		& -μμH		& -μμH 		& -μμH		& -μμH		&	&\\
			&   &   & ✓ &   &			& g̱-		&	& -μμH		& -μμH		& -μμH		& -μμH		& -μμH		&	&\\
			&   &   &   & ✓ &			& g-		&	& -μμH		& -μμH		& -μμH		& -μμH		& -μμH		&	&\\
\bottomrule
\end{tabular}
\caption{Aspect morphology: consecutive (realizational subordinate clause) aspect \textit{\xx{cnj}-}}
\label{tab:aspect-morphology-csec}
\end{table}

\clearpage
\subsection{Conditional and contingent}\label{sec:asp-condctng}

\begin{multicols}{2}
\noindent
\end{multicols}

\clearpage
\begin{table}
\centerfloat
\begin{tabular}{l
		c@{\hspace{1ex}}c@{\hspace{1ex}}c@{\hspace{1ex}}c
		rrr
		*{5}{l}ll}
\toprule
\textit{Aspect}		& \textit{∅}
			    & \textit{n}
			        & \textit{g̱}
			            & \textit{g}
					& \textit{Preverbs}	& \textit{Asp.\ pfxs.}
										& \textit{Stv.}
											& \rt{CV}	& \rt{CVʰ}	& \rt{CVC}	& \rt{CVCʼ}	& \rt{CVʼC}	& \textit{Suffixes}	
																						& \textit{Notes}\\
\midrule
consecutive		& ✓ &   &   &   &			&		&	& -μͤμH		& -μͤμH		& -μH		& -μH		& -μH		& -n-í	&\\
			&   & ✓ &   &   &			& n-		&	& -μͤμH		& -μͤμH		& -μH 		& -μH		& -μH		& -n-í	&\\
			&   &   & ✓ &   &			& g̱-		&	& -μͤμH		& -μͤμH		& -μH		& -μH		& -μH		& -n-í	&\\
			&   &   &   & ✓ &			& g-		&	& -μͤμH		& -μͤμH		& -μH		& -μH		& -μH		& -n-í	&\\
\bottomrule
\end{tabular}
\caption{Aspect morphology: conditional (subordinate clause) mood \textit{\xx{cnj}-} … \textit{-n} + \textit{-í}}
\label{tab:aspect-morphology-cond}
\end{table}

\begin{table}
\centerfloat
\begin{tabular}{l
		c@{\hspace{1ex}}c@{\hspace{1ex}}c@{\hspace{1ex}}c
		rrr
		*{5}{l}ll}
\toprule
\textit{Aspect}		& \textit{∅}
			    & \textit{n}
			        & \textit{g̱}
			            & \textit{g}
					& \textit{Preverbs}	& \textit{Asp.\ pfxs.}
										& \textit{Stv.}
											& \rt{CV}	& \rt{CVʰ}	& \rt{CVC}	& \rt{CVCʼ}	& \rt{CVʼC}	& \textit{Suffixes}	
																						& \textit{Notes}\\
\midrule
contingent		& ✓ &   &   &   &			&		&	& -μͤμH		& -μͤμH		&		&		&		& -n-ín	&\\
			& ✓ &   &   &   &			&		&	&		&		& -μH		& -μH		& -μH		& \phantom{-n}-ín	&\\
\addlinespace[0.25em]
			&   & ✓ &   &   &			& n-g̱-		&	& -μͤμH		& -μͤμH		& 		&		&		& -n-ín	&\\
			&   & ✓ &   &   &			& n-g̱-		&	&		&		& -μH 		& -μH		& -μH		& \phantom{-n}-ín	&\\
\addlinespace[0.25em]
			&   &   & ✓ &   &			& g̱-g̱-		&	& -μͤμH		& -μͤμH		&		&		&		& -n-ín	&\\
			&   &   & ✓ &   &			& g̱-g̱-		&	&		&		& -μH		& -μH		& -μH		& \phantom{-n}-ín	&\\
\addlinespace[0.25em]
			&   &   &   & ✓ &			& g-g̱-		&	& -μͤμH		& -μͤμH		&		&		&		& -n-ín	&\\
			&   &   &   & ✓ &			& g-g̱-		&	&		&		& -μH		& -μH		& -μH		& \phantom{-n}-ín	&\\
\bottomrule
\end{tabular}
\caption{Aspect morphology: contingent (subordinate clause) modality \textit{\xx{cnj}-} + \textit{g̱-} … \textit{-n} + \textit{-ín}}
\label{tab:aspect-morphology-ctng}
\end{table}
