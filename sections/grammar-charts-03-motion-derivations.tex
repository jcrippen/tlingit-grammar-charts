%!TEX root = ../grammar-charts.tex
%%
%% Motion derivations.
%%

\clearpage
\section{Motion derivations}\label{sec:motion-derivations}

\begin{multicols}{2}
\noindent
Motion derivations are derivational structures that describe a path, location, direction, or manner of motion.
The path, direction, location, or manner supplied by a motion derivation is termed a ‘path argument’ although not all denote paths in space.
The path argument may consist of one or more preverbs or a PP or both.
If the path argument contains a PP this usually includes an unspecified ‘variable’ NP which must be filled by a pronoun or phrase and which denotes the location to which the path argument is anchored.
The variable NP may be the complement of the postposition or it may occur inside of another NP that is the complement of the postposition.
The latter case typically has the variable NP as the possessor of an inalienable spatial noun such as \fm{daa} ‘around, about, surrounding’.

Although the path argument is the most obvious feature of a motion derivation, they supply more than just the path argument.
In particular, they always supply a conjugation class which replaces any existing conjugation class of the verb.
Motion derivations also supply a repetitive imperfective aspect form.
In addition they may supply verb prefixes such as an IN (§\ref{sec:incorporates}), an irrealis prefix, or the middle voice \fm{d-} prefix.

The conjugation class specified by a motion derivation is spatially significant.
The \textbf{\fm{n}-conjugation class} motion derivations all describe motion along a plane oriented horizontally with respect to the origo.
Along this horizontal plane the motion derivation describes a direction which may be forward, backward, leftward, rightward, or at an angle with respect to the origo.
The \textbf{\fm{g̱}-conjugation class} motion derivations all describe motion downward with respect to the origo.
This downward motion may or may not include a horizontal component, but this has no effect on the conjugation class specification.
The \textbf{\fm{g}-conjugation class} motion derivations all describe motion upward with respect to the origo.
This upward motion also may or may not include a horizontal component which is irrelevant to the conjugation class.
The \textbf{\fm{∅}-conjugation class} has no particular spatial orientation.
Thus motion derivations in this class may describe motion in any horizontal or vertical direction or a combination of them.
The \fm{∅}-conjugation class can be thought of as the spatially unspecified elsewhere case.

\FIXME{Discuss postpositions.}

\FIXME{Discuss preverbs.}

\FIXME{Discuss repetitives.}

\FIXME{Describe layout of charts.}
\end{multicols}

\clearpage
\subsection{\textit{N-}, \textit{g̱-}, and \textit{g}-conjugation motion derivations}

\clearpage
\begin{table}
\centerfloat
\begin{tabular}{rrclll}
\toprule
\textit{Path arg.}	& \textit{Prefixes}	& \textit{Conj.}	& \textit{Repetitive}	& \textit{Translation}			& \textit{Notes}\\
\midrule
			&			& n		& yoo= … i- … -k	& along laterally, horizontally		&\\
NP-x̱			&			& n		& yoo= … i- … -k	& along NP				& with \fm{-x̱} pertingent\\
NP-dé			&			& n		& yoo= … i- … -k	& toward NP				& with \fm{-dé} allative\\
NP-dáx̱			&			& n		& yoo= … i- … -k	& away from NP				& with \fm{-dáx̱} ablative\\
NP-náx̱			&			& n		& yoo= … i- … -k	& via, along, across, through NP	& with \fm{-náx̱} perlative\\
NP-náḵ			&			& n		& yoo= … i- … -k	& leaving NP behind			& with \fm{-náḵ} abessive\\
NP-g̱áa			&			& n		& yoo= … i- … -k	& going to obtain NP; near NP		& with \fm{-g̱áa} adessive\\
\addlinespace[0.75em]
yux̱\≠			&			& n		& yoo= … i- … -k	& out of house				& from \fm{yú} \~\ \fm{yóo} distal + \fm{-x̱} pertingent\\
NP-xʼ yux̱\≠		&			& n		& yoo= … i- … -k	& out of house at NP			& with \fm{-xʼ} locative\\
\addlinespace[0.75em]
NP-t			&			& n		& yoo= … i- … -k	& around, circling, perambulating NP	& with \fm{-t} punctual\\
áa			&			& n		& yoo= … i- … -k	& around and about			& from \fm{á} ‘it, there’ + \fm{-μ} locative\\
\bottomrule
\end{tabular}
\caption{Motion derivations assigning \textit{n}-conjugation class}
\label{tab:motion-derivations-n}
\end{table}

\clearpage
\begin{table}
\centerfloat
\begin{tabular}{rrclll}
\toprule
\textit{Path arg.}	& \textit{Prefixes}	& \textit{Conj.}	& \textit{Repetitive}	& \textit{Translation}			& \textit{Notes}\\
\midrule
			&			& g̱		& yei= … -ch		& downward, falling			&\\
\addlinespace[0.75em]
ÿaa\≠			&			& g̱		& yei= … -ch		& down along				& \fm{yei=} in repetitive blocks \fm{ÿaa=}\\
\addlinespace[0.75em]
NP-x̱			&			& g̱		& yei= … -ch		& down along NP				& with \fm{-x̱} pertingent\\
héen-x̱			&			& g̱		& yei= … -ch		& down into water			& with \fm{héen} ‘water’\\
ká-x̱			& sha-			& g̱		& yei= … -ch		& falling over, lying prone		& with \fm{ká} ‘horizontal surface’\\
yaax̱\≠			&			& g̱		& yei= … -ch		& embarking, aboard boat, into vehicle	& from \fm{yaakw} ‘boat’ + \fm{-x̱}\\
\addlinespace[0.75em]
NP-náx̱			&			& g̱		& yei= … -ch		& down via, through NP			& with \fm{-náx̱} perlative\\
ÿanax̱\≠			&			& g̱		& yei= … -ch		& down into ground			& from \fm[*]{ŋən} > \fm[*]{ÿán} ‘ground’ + \fm{-náx̱} perlative\\
\bottomrule
\end{tabular}
\caption{Motion derivations assigning \textit{g̱}-conjugation class}
\label{tab:motion-derivations-gh}
\end{table}

\clearpage
\begin{table}
\centerfloat
\begin{tabular}{rrclll}
\toprule
\textit{Path arg.}	& \textit{Prefixes}	& \textit{Conj.}	& \textit{Repetitive}	& \textit{Translation}			& \textit{Notes}\\
\midrule
			&			& g		& kei= … -ch		& upward, starting, picking up		&\\
NP-dáx̱			&			& g		& kei= … -ch		& upward, starting, picking up from NP	& with \fm{-dáx̱} ablative\\
\addlinespace[0.75em]
ḵut\≠			&			& g		& —			& going astray, getting lost		& from \fm[*]{ḵú} areal + \fm{-t}\\
\bottomrule
\end{tabular}
\caption{Motion derivations assigning \textit{g}-conjugation class}
\label{tab:motion-derivations-g}
\end{table}

\clearpage
\subsection{\textit{∅}-conjugation motion derivations}

\clearpage
\begin{table}
\centerfloat
\begin{tabular}{rrclll}
\toprule
\textit{Path arg.}	& \textit{Prefixes}	& \textit{Conj.}	& \textit{Repetitive}	& \textit{Translation}			& \textit{Notes}\\
\midrule
NP-\{t,x̱,dé\}		&			& ∅		& -μμL			& arriving at NP			& rep.\ \fm{-x̱}, prosp.\ \&\ prog.\ \fm{-dé}\\
yóo-\{t,x̱,dé\}		&			& ∅		& -μμL			& off, somewhere, away			& with \fm{yú} distal deictic\\
\addlinespace[0.75em]
ÿan= \~\ ÿax̱= \~\ ÿán-de\≠ &			& ∅		& -μμL			& ashore, finishing, terminating	& with \fm{ÿán} ‘shore’\\
ÿan= \~\ ÿax̱= \~\ ÿán-de\≠ & kʼi-		& ∅		& -μμL			& setting up, erecting			& with \fm{kʼi-} ‘base’\\
ÿan= \~\ ÿax̱= \~\ ÿán-de\≠ & sha-		& ∅		& -μμL			& leaning against			& with \fm{sha-} ‘head’\\
NP-xʼ ÿan= \~\ ÿax̱= \~\ ÿán-de\≠ &		& ∅		& -μμL			& coming to rest at NP			& with \fm{-xʼ} locative\\
NP-náx̱ ÿan= \~\ ÿax̱= \~\ ÿán-de\≠ &		& ∅		& -μμL			& across NP, to other side of NP	& with \fm{-náx̱} perlative\\
\addlinespace[0.75em]
haa-t= \~\ haa-x̱= \~\ haa(n)-dé\≠ &		& ∅		& -μμL			& here, hither, toward speaker		& with \fm{haaⁿ} ‘here’\\
\addlinespace[0.75em]
neil(-t)= \~\ neil-x̱= \~\ neil-dé\≠ &		& ∅		& -μμL			& home, inside (of building)		& with \fm{neil} ’home’\\
NP-xʼ neil(-t)= \~\ neil-x̱= \~\ neil-dé\≠ &	& ∅		& -μμL			& inside at NP				& with \fm{-xʼ} locative\\
\addlinespace[0.75em]
kúx= \~\ kúx-x̱= \~\ kúx-de\≠ &			& ∅		& -μμL			& aground, into shallow water		& with \fm{\rt{kux}} ‘dry’\\
\bottomrule
\end{tabular}
\caption{Motion derivations assigning \textit{∅}-conjugation class – postpositions \textit{-t} \~\ \textit{-x̱} \~\ \textit{-dé} with \fm{-μμL} repetitive}
\label{tab:motion-derivations-zero-txhde}
\end{table}

\clearpage
\begin{table}
\centerfloat
\begin{tabular}{rrclll}
\toprule
\textit{Path arg.}	& \textit{Prefixes}	& \textit{Conj.}	& \textit{Repetitive}	& \textit{Translation}			& \textit{Notes}\\
\midrule
kei\≠			&			& ∅		& -ch			& up					&\\
ux̱=kei\≠		&			& ∅		& -ch			& out of control, blindly, amiss	&\\
NP x̱ʼé-μ kei\≠		&			& ∅		& -ch			& catching up with NP			& with \fm{x̱ʼé} ‘mouth’ + locative \fm{-μ}\\
\addlinespace[0.75em]
yei\≠			&			& ∅		& -ch			& disembarking, exiting vehicle		&\\
\addlinespace[0.75em]
yeiḵ\≠			&			& ∅		& -ch			& down to beach from land		& from \fm{éeḵ} \~\ \fm{éiḵ} ‘beach’\\
héen-i yeiḵ\≠		&			& ∅		& -ch			& from shore into water			& with \fm{-í} locative\\
\addlinespace[0.75em]
daaḵ\≠			&			& ∅		& -ch			& inland, back from open, off fire	& from \fm{dáaḵ} ‘inland’\\
dáag̱-i daaḵ\≠		&			& ∅		& -ch			& further inland			& with \fm{-í} locative\\
ḵwáaḵ-x̱ daaḵ\≠		&			& ∅		& -ch			& by mistake, wrongly			& with \fm{ḵwáaḵ} ‘mistake’\\
\addlinespace[0.75em]
daak\≠			&			& ∅		& -ch			& seaward, into open, onto fire, falling & from \fm{dáak} ‘seaward’\\
\addlinespace[0.75em]
ḵux̱= \~\ ḵúx̱-de\≠	& d-			& ∅		& -ch			& going back, returning directly	& from \fm{ḵú} areal + \fm{-x̱} pertingent\\
NP-xʼ ḵux̱= \~\ ḵúx̱-de\≠	& d-			& ∅		& -ch			& returning to NP			& with \fm{-xʼ} locative\\
\bottomrule
\end{tabular}
\caption{Motion derivations assigning \textit{∅}-conjugation class – directional preverbs with \textit{-ch} repetitive}
\label{tab:motion-derivations-zero-ch}
\end{table}

\clearpage
\begin{table}
\centerfloat
\begin{tabular}{rrclll}
\toprule
\textit{Path arg.}	& \textit{Prefixes}	& \textit{Conj.}	& \textit{Repetitive}	& \textit{Translation}			& \textit{Notes}\\
\midrule
g̱unéi= \~\ g̱unayéi\≠	&			& ∅		& -x̱			& beginning, starting			& from \fm{g̱una} ‘other’ + \fm{yé} ‘way, place’\\
\addlinespace[0.75em]
NP-xʼ			&			& ∅		& -x̱			& nearing NP				& with \fm{-xʼ} locative\\
NP ÿá-μ			&			& ∅		& -x̱			& approaching NP			& with \fm{ÿá} ‘face’\\
NP g̱una-ÿá-μ		&			& ∅		& -x̱			& separating from NP			& with \fm{g̱una} ‘other’\\
NP ji-shá-μ		&			& ∅		& -x̱			& getting ahead of NP			& with \fm{jín} ‘hand’ + \fm{shá} ‘head’\\
\addlinespace[0.75em]
NP-xʼ ÿáx̱\≠		&			& ∅		& -x̱			& turning over by NP			& with \fm{ÿáx̱=} ‘facing’\\
á-μ ÿáx̱\≠		&			& ∅		& -x̱			& turning over				& with \fm{á} ‘there’\\
shú-μ ÿáx̱\≠		&			& ∅		& -x̱			& turning end over end			& with \fm{shú} ‘end’\\
\addlinespace[0.75em]
gág-i			&			& ∅		& -x̱			& emerging, out into open		& from \fm{gáak} ‘protrusion’\\
dáag̱-i			&			& ∅		& -x̱			& out of water				& from \fm{dáaḵ} ‘inland’\\
héen-i			&			& ∅		& -x̱			& into water				& with \fm{héen} ‘water’\\
éeg̱-i \~\ éig̱-i		&			& ∅		& -x̱			& to beach				& with \fm{éeḵ} \~\ \fm{éiḵ} ‘beach’\\
\addlinespace[0.75em]
NP-x̱			&			& ∅		& -x̱			& stuck at NP, in place at NP		& with \fm{-x̱} pertingent\\
\addlinespace[0.75em]
ÿetx̱= \~\ ÿedax̱\≠	&			& ∅		& -x̱			& starting, taking off			& from \fm[*]{ÿé} ‘place’ + \fm{-dáx̱} ablative\\
\bottomrule
\end{tabular}
\caption{Motion derivations assigning \textit{∅}-conjugation class – locative \textit{-xʼ}, \textit{-μ}, or \fm{-í} with \textit{-x̱} repetitive}
\label{tab:motion-derivations-zero-xh}
\end{table}

%\clearpage
\begin{table}
\centerfloat
\begin{tabular}{rrclll}
\toprule
\textit{Path arg.}	& \textit{Prefixes}	& \textit{Conj.}	& \textit{Repetitive}	& \textit{Translation}			& \textit{Notes}\\
\midrule
yoo\≠			&			& ∅		& yoo= … i- … -k	& back and forth, to and fro		&\\
ÿan=yoo\≠		&			& ∅		& yoo= … i- … -k	& up and down (from surface)		& from \fm[*]{ŋən} > \fm[*]{ÿán} ‘ground’\\
\bottomrule
\end{tabular}
\caption{Motion derivations assigning \textit{∅}-conjugation class – alternating \fm{yoo=} with \textit{yoo= … i- … -k} repetitive}
\label{tab:motion-derivations-zero-yoo}
\end{table}

\clearpage
\begin{table}
\centerfloat
\begin{tabular}{rrclll}
\toprule
\textit{Path arg.}	& \textit{Prefixes}	& \textit{Conj.}	& \textit{Repetitive}	& \textit{Translation}			& \textit{Notes}\\
\midrule
NP-xʼ ÿaa\≠		& \~\ ÿa-uˑ-		& ∅		& -ch			& obliquely, circuitously at NP		& with \fm{-xʼ} locative\\
NP daséi-xʼ ÿaa\≠	& \~\ ÿa-uˑ-		& ∅		& -ch			& exchanging places with NP		& with \fm{daséi} ‘reach’\\
\addlinespace[0.75em]
NP-x̱ ÿaa\≠		& \~\ ÿa-uˑ-		& ∅		& -ch			& obliquely, circuitously along NP	& with \fm{-x̱} pertingent\\
NP daa-x̱ ÿaa\≠		& \~\ ÿa-uˑ-		& ∅		& -ch			& circling around NP			& with \fm{daa} ‘around’\\
\addlinespace[0.75em]
NP-dé ÿaa\≠		& \~\ ÿa-uˑ-		& ∅		& -ch			& obliquely, circuitously toward NP	& with \fm{-dé} allative\\
héi-de ÿaa\≠		& \~\ ÿa-uˑ-		& ∅		& -ch			& aside, out of the way			& with \fm{hé} mesioproximal\\
\addlinespace[0.75em]
NP-dáx̱ ÿaa\≠		& \~\ ÿa-uˑ-		& ∅		& -ch			& obliquely, circuitously away from NP	& with \fm{-dáx̱} ablative\\
NP ji-kaa-dáx̱ ÿaa\≠	& \~\ ÿa-uˑ-		& ∅		& -ch			& out of NP’s way			& with \fm{jín} ‘hand’ + \fm{ká} ‘horiz.\ sfc.’\\
\addlinespace[0.75em]
NP-náx̱ ÿaa\≠		& \~\ ÿa-uˑ-		& ∅		& -ch			& obliquely, circuitously through NP	& with \fm{-náx̱} perlative\\
\bottomrule
\end{tabular}
\caption{Motion derivations assigning \textit{∅}-conjugation class – oblique \fm{ÿaa=} (rep.)\ \~\ \fm{ÿa-uˑ-} with \textit{-ch} repetitive}
\label{tab:motion-derivations-zero-yau}
\end{table}

%\clearpage
\begin{table}
\centerfloat
\begin{tabular}{rrclll}
\toprule
\textit{Path arg.}	& \textit{Prefixes}	& \textit{Conj.}	& \textit{Repetitive}	& \textit{Translation}			& \textit{Notes}\\
\midrule
NP-x̱			& sha-ÿa-uˑ-		& ∅		& -ch			& hanging up at NP			& with \fm{sha-} ‘head’\\
ÿax̱\≠			& sha-ÿa-uˑ-		& ∅		& -ch			& hanging up				& with \fm{ÿáx̱=} ‘facing’\\
\bottomrule
\end{tabular}
\caption{Motion derivations assigning \textit{∅}-conjugation class – hanging \fm{sha-ÿa-uˑ-} with \textit{-ch} repetitive}
\label{tab:motion-derivations-zero-shayau}
\end{table}

%\clearpage
\begin{table}
\centerfloat
\begin{tabular}{rrclll}
\toprule
\textit{Path arg.}	& \textit{Prefixes}	& \textit{Conj.}	& \textit{Repetitive}	& \textit{Translation}			& \textit{Notes}\\
\midrule
			& a-ÿa-uˑ-d-		& ∅		& -x̱			& reverting, returning circuitously	& expletive \fm{a-} + middle voice \fm{d-}\\
\bottomrule
\end{tabular}
\caption{Motion derivations assigning \textit{∅}-conjugation class – perambulative revertive \fm{a-ÿa-uˑ-d} with \textit{-x̱} repetitive}
\label{tab:motion-derivations-zero-ayaud}
\end{table}
